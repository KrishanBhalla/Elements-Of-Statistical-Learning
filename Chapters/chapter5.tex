
\chapter{Chapter 5 - Basis Expansions and Regularisation}

\subsection*{5.1 Show that the truncated power basis functions represent a basis for a cubic spline with the two knots as indicated}

We'll prove a more general result for $p$ knots, $\xi_1, \dots, \xi_p$
The basis described for these knots is:

\begin{align*}
    h_1(X) &= 1 \\
    h_2(X) &= X \\
    h_3(X) &= X^2 \\
    h_4(X) &= X^3 \\
    h_{4+i}(X) &= (X-\xi_i)^3_+ \hspace{2em} \forall i \in \{1,\dots,p\} \\
\end{align*}

We must show that given any cubic spline with knots at these places, it can be represented by a linear combination of the above and the representation is unique.

With $\xi_0 = -\infty$ and $\xi_{p+1} = \infty$, suppose that we have some piece-wise cubic spline defined by: 
$$f(X) = \sum_{i=0}^p f_i(X)\mathbb{1}_{[\xi_i,\xi_{i+1}]}$$
With $f_i(\xi_{i+1}) = f_{i+1}(\xi_{i+1})$, $0 \leq i \leq p$ and similar for first and second derivatives, each $f_i$ being a cubic.
Then let $g_i = f_{i+1} - f_i$, and we have $g_i(\xi_{i+1})  = 0$,  $g'_i(\xi_{i+1})  = 0$,  $g''_i(\xi_{i+1})  = 0$

$g_i$ is a cubic, and so it is clear that $g_i$ must take the form $a_i(X - \xi_{i+1})^3$ for some $a_i$. Adding in our indicator functions:
\begin{align*}
    & f_i(X)\mathbb{1}_{[\xi_i,\xi_{i+1}]} + f_{i+1}(X)\mathbb{1}_{[\xi_{i+1},\xi_{i+2}]} \\
    &=  f_{i}(X)\mathbb{1}_{[\xi_i,\xi_{i+2}]} + a_i(X - \xi_{i+1})^3\mathbb{1}_{[\xi_{i+1},\xi_{i+2}]}
\end{align*}
Applying this to each knot, we get:

\begin{align*}
    f(X) &= c_1 + c_2 X + c_3 X^2 + c_4 X^4 \\
    &+ \sum_{i=0}^p a_i (X - \xi_{i+1})^3 \mathbb{1}_{[\xi_i,\xi_{i+1}]}
\end{align*}

For some constants $c_i$. Recasting this slightly gives
$$f(X) = \sum_{i=0}^{p+4} b_i h_i(X)$$ 
For some $b_i$ as required.

For linear independence, note that all knots are distinct, and so if any of the $h_i$, $i>4$ can be formed from a linear combination of the others, then left of $\xi_i$ this combination must be zero. However this is an open set and a polynomial either has finitely many roots or is identically zero. Thus we have that the $h_i$ form a basis as required. 


\subsection*{5.4 Consider the truncated power series representation for cubic splines with K interior knots. Let
$$f(X) = \sum_{j=0}^3 \beta_j X^j + \sum_{k=1}^K \theta_k (X - \xi_k)^3_+$$
Prove that the natural boundary conditions for the natural cubic splines (Section 5.2.1) imply the following linear constraints on the coefficients:
\begin{align*}
    \beta_2 &= 0 \\
    \beta_3 &= 0 \\
    \sum_{k=1}^K \theta_k &= 0 \\
    \sum_{k=1}^K \theta_k \xi_k  &= 0
\end{align*}
Hence derive the basis (5.4) and (5.5)
}

$f(X)$ is constrained to be linear for $X \leq \xi_1$ and $X \geq \xi_k$.
On the left hand side, this implies that $\beta_2$ and $\beta_3$ are $0$.
On the right hand side, look at coefficients of the polynomial.
The $X^3$ term has coefficient $\sum_{k=1}^K \theta_k$, and the $X^2$ term has coefficient $\sum_{k=1}^K \theta_k \xi_k $, which must then both be $0$.

Thus we know that we can remove $h_2$ and $h_3$ in our basis from Ex 5.1.

Consider $N_{k+2}(X) = d_k(X) - d_{K-1}(X)$ in the book (5.4). Here the $d_k$ are as in (5.5).
$$N_{k+2}(X) = \frac{(X - \xi_k)^3_+ - (X - \xi_K)^3_+}{\xi_k - \xi_K} - \frac{(X - \xi_{K-1})^3_+ - (X - \xi_K)^3_+}{\xi_{K-1} - \xi_K}$$

We want to write $\sum_{k=1}^K \theta_k (X - \xi_k)^3_+$ in terms of the $N_{k+2}$
We will try with some coefficients $a_k$, and see if we can find a solution $\sum_{k=1}^K \theta_k (X - \xi_k)^3_+ = \sum_{k=1}^{K-2} a_k \theta_k N_{k+2}(X)$
Then:
\begin{align*}
    \sum_{k=1}^{K-2}a_k \theta_k N_{k+2}(X) &= \sum_{k=1}^{K-2 }a_k \theta_k \left( d_k(X) - d_{K-1}(X) \right) \\
    &= \sum_{k=1}^{K-2 }a_k \theta_k \left( \frac{(X - \xi_k)^3_+ - (X - \xi_K)^3_+}{\xi_k - \xi_K} \right) \\ &- \left(\sum_{k=1}^{K-2 }a_k \theta_k \right) \frac{(X - \xi_{K-1})^3_+ - (X  - \xi_K)^3_+}{\xi_{K-1} - \xi_K}
\end{align*}

To eliminate the denominator, let's try $a_k$ = $\xi_k - \xi_K$

\begin{align*}
    \sum_{k=1}^{K-2}a_k \theta_k N_{k+2}(X)  &=  \sum_{k=1}^{K-2 }\theta_k \left( (X - \xi_k)^3_+ - (X - \xi_K)^3_+ \right) \\ 
    &- \left(\sum_{k=1}^{K-2 }(\xi_k - \xi_K) \theta_k \right) \frac{(X - \xi_{K-1})^3_+ - (X  - \xi_K)^3_+}{\xi_{K-1} - \xi_K} \\
    &=  \sum_{k=1}^{K-2 }\theta_k \left( (X - \xi_k)^3_+ - (X - \xi_K)^3_+ \right)\\
    &+ (\xi_{K-1}\theta_{K-1} - \xi_{K}\theta_{K-1})  \frac{(X - \xi_{K-1})^3_+ - (X  - \xi_K)^3_+}{\xi_{K-1} - \xi_K}\\
    &=  \sum_{k=1}^{K-2 }\theta_k (X - \xi_k)^3_+  + (\theta_{K-1} + \theta_K) (X - \xi_K)^3_+\\
    &+ (\xi_{K-1}- \xi_{K})\theta_{K-1} \frac{(X - \xi_{K-1})^3_+ - (X  - \xi_K)^3_+}{\xi_{K-1} - \xi_K}\\
    &=  \sum_{k=1}^{K-2 }\theta_k (X - \xi_k)^3_+  + (\theta_{K-1} + \theta_K) (X - \xi_K)^3_+\\
    &+ \theta_{K-1} (X - \xi_{K-1})^3_+ - \theta_{K-1}(X  - \xi_K)^3_+\\
    &=  \sum_{k=1}^{K-2 }\theta_k (X - \xi_k)^3_+  + \theta_K (X - \xi_K)^3_+\\
    &+ \theta_{K-1} (X - \xi_{K-1})^3_+\\
    &=  \sum_{k=1}^{K }\theta_k (X - \xi_k)^3_+
\end{align*}
Where use our $\theta$ conditions from above.
Thus by setting $a_k = \xi_k - \xi_K$ we can in fact retrieve $f(X)$, and so we have $K$ functions from which any element of the space can be generated. This space has dimension $K$ and so (5.4) must define a basis as required.

\subsection*{5.7 Derivation of smoothing splines. Please see the book. $g$ is a natural cubic spline interpolating $N$ knots $x_i$ $i \in \{1,\dots,N\}$, with values $g(x_i) = z_i$.
$a$ and $b$ are s.t. $a < x_1 < \dots x_N < b$.}
\subsection*{a) Let $h$ is the difference between $g$ and another differentiable function that interpolates the pairs $(x_i, z_i)$.}

\begin{align*}
    \int_a^b g''(x)h''(x) dx &= [g''(x)h'(x)]_a^b - \int_a^b g'''(x) h'(x) dx \\
    &= [0 \cdot h'(b) - 0 \cdot h'(b)] - \int_{x_1}^{x_N} g'''(x) h'(x) dx \\
    &= -\sum_{j = 1}^{N-1} \int_{x_j}^{x_{j+1}} g'''(x) h'(x) dx \\
    \int_{x_j}^{x_{j+1}} g'''(x) h'(x) dx &=  [g'''(x)h(x)]^{x_{j+1}}_{x_j}  - \int_{x_j}^{x_{j+1}} g''''(x) h(x) dx \\
    &=  [g'''(x)h(x)]^{x_{j+1}}_{x_j}  \\
    \Rightarrow  \int_a^b g''(x)h''(x) dx &= -\sum_{j = 1}^{N-1} [g'''(x)h(x)]^{x_{j+1}}_{x_j} \\
    &= 0
\end{align*}
The last line following as $h(x_i) = 0$ $\forall i$


\subsection*{b) Show that $\int_a^b \tilde{g}''(x)^2  dx \geq \int_a^b g''(x)^2 dx$ with equality only when $h = 0$ on $[a,b]$}
\begin{align*}
    0 &= \int_a^b \tilde{g}''(x)^2  dx \\
    &= \int_a^b \left(g''(x) + h''(x)\right)^2 dx \\
    &= \int_a^b g''(x)^2  + 2 * g''(x) h''(x) + h''(x)^2 dx\\
    &= \int_a^b g''(x)^2 dx +  \int_a^b  h''(x)^2 dx \\
    &\geq \int_a^b g''(x)^2 dx
\end{align*}
With the penultimate line following from a), and the final line being an equality only if $ h''$ is identically 0 in $[a,b]$. That is to say $h$ is linear in this interval, but at each $x_i$ h is $0$, and so given at least 2 $x_i$, the last line holds if and only if $h$ is identically $0$ in $[a,b]$ 


\subsection*{c) Consider the penalised least squares problem:
$$ \min_f \left[ \sum_{i=1}^N \left(y_i - f(x_i)\right)^2 + \lambda \int_a^b f''(t)^2 dt \right]$$
Show that the minimiser must be a cubic spline with knots at each of the $x_i$}

Let $\tilde{f}$ be the minimiser. let $z_i = \tilde{f}(x_i)$ for every $i$. 
Then we can find a natural cubic spline interpolating these points, $g$ say.
Thus $\sum_{i=1}^N \left(y_i - \tilde{f}(x_i)\right)^2 = \sum_{i=1}^N \left(y_i - g(x_i)\right)^2$

From b), we know that $\int_a^b \tilde{f}''(t)^2 dt \geq \int_a^b g''(t)^2 dt$, and so $f$ can only be the minimiser if these values are in fact equal.
Again from b), this only occurs if $h$ is identically $0$ in $[a,b]$, so $\tilde{f} = g$ in $[a,b]$, and the minimiser is a natural cubic spline.


\subsection*{5.9 Derive the Reinsch form $S_\lambda = \left(I + \lambda K\right) ^{-1}$ for the smoothing spline.}

We know $S_\lambda = N\left(N^T N + \lambda \Omega_N\right)^{-1}N^T$
We have a basis as in Ex 5.4, of the form $N_1(X) = 1$, $N_2(X) = X$, and $N_{k+2}(X) = d_k(X) - d_{K-1}(X)$ for $k <= N-2$ where we have $N$ knots. Let the knots be $x_1 < \dots < x_n$
$d_j(x_i) = 0$ if $j \geq i$, and so our matrix $N$, which has $N_{i,j} = N_j(x_i)$ has all but the first two columns $0$ above the diagonal \footnote{$N_j(x_i) = 0 $ for $i < j$ and $j >= 2$}.
The first column is all $1$s and the second a column of $x_i$.
$$
N = \begin{bmatrix}
1 & x_1 & 0 & \dots & 0\\
1 & x_2 & 0 & \dots & 0\\
1 & x_3 & N_3(x_3) & \dots & 0\\
\dots & \dots & \dots & \dots & \dots\\
1 & x_N & N_3(x_N) & \dots & N_N(x_N)\\
\end{bmatrix}
$$

We can subtract the first row form the rest to get:
$$
N = \begin{bmatrix}
1 & 0 & 0 & \dots & 0\\
0 & x_2 - x_1 & 0 & \dots & 0\\
0 & x_3 - x_1 & N_3(x_3) & \dots & 0\\
\dots & \dots & \dots & \dots & \dots\\
0 & x_N - x_1 & N_3(x_N) & \dots & N_N(x_N)\\
\end{bmatrix}
$$

Now the determinant of $N$ is the determinant of the submatrix consisting of $N$ less the first row and column.
This submatrix is lower triangular with a non-zero diagonal (as the $x_i$ are distinct).
Hence $N$ is invertible.

Then:
\begin{align*}
S_\lambda &= N\left(N^T N + \lambda \Omega_N\right)^{-1}N^T  \\
&= \left((N^T)^{-1} \left( N^T N + \lambda \Omega_N \right) N^{-1}\right)^{-1} \\
&= \left((N^T)^{-1} N^T N N^{-1} + \lambda (N^T)^{-1} \Omega_N  N^{-1}\right)^{-1} \\
&= \left(I + \lambda K \right)^{-1} 
\end{align*}
Where $K = (N^T)^{-1} \Omega_N  N^{-1}$.

\subsection*{5.12 
Characterise the solution to the following problem:
$$ min_f RSS(f, \lambda) = \sum_{i=1}^N w_i \left( y_i - f(x_i)\right)^2 + \lambda \int f''(t)^2 dt $$
Where the $w_i \geq 0$ are the observation weights. Characterise the solution to the smoothing spline problem when the training data have ties in $X$}

We know that the solution is a natural spline, and will reform the $RSS$ as:
$$RSS(f, \theta) = \left(Y - N\theta \right)^T W \left(Y - N\theta\right) + \lambda \theta^T \Omega_N \theta$$
$W$ being a diagonal matrix with $(W)_{i,i} = w_i$.
Setting the first derivative to zero we get:
\begin{align*}
-2 N^T W \left(Y - N\theta\right) + 2 \lambda \Omega_N \theta &= 0\\
\lambda \Omega_N \theta  &= N^T W \left(Y - N\theta\right)\\
\left( N^T W N + \lambda \Omega_N\right) \theta &= N^T W Y \\
 \theta &= \left( N^T W N + \lambda \Omega_N\right)^{-1} N^T W Y 
\end{align*}
This is simply the solution to a weighted ridge regression.

Using the Reinsch form, we end up with $S_\lambda = \left( W + \lambda K \right)^{-1}$.
In the smoothing spline problem with ties, given $K$ values for $x_i$, of which $L$ are unique, we can arrive at the weighted regression above on $L$ parameters (knots), and $N$ having dimension $L \times L$.
