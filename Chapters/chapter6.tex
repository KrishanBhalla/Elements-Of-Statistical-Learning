
\chapter{Chapter 6 - Kernel Smoothing Methods}


\subsection*{6.1 Show that the Nadaraya-Watson kernel weighted average with fixed metric bandwidth $\lambda$ and a Gaussian kernel is differentiable. What can be said for the Epanechnikov kernel? What can be said for the Epanechnikov kernel with adaptive nearest-neighbour bandwidth $\lambda(x_0)$}

Nadaraya-Watson:
$$ \hat{f}(x_0) = \frac{\sum_{i=1}^N K_\lambda(x_0,x_i)y_i}{\sum_{i=1}^N K_\lambda(x_0,x_i)}$$
Gaussian Kernel:
$$ K_\lambda(x_0, x) = \frac{1}{\lambda \sqrt{2\pi}} \exp\left(-\frac{(x - x_0)^2}{2\lambda^2}\right) $$

Thus $K_\lambda(x_0, x)$ is strictly-positive and differentiable (in $x_0$ say) everywhere, so our denominator is too. Similarly the numerator is a linear combination of smooth functions so is smooth.
Thus the Nadaraya-Watson kernel weighted average is differentiable everywhere\footnote{A ratio of smooth functions such that the denominator has no zeros is differentiable} as a function of $x_0$.

The Epanechnikov kernel is defined as:
\begin{align*}
    K_\lambda(x_0, x) &= D\left(\frac{\abs{x - x_0}}{\lambda}\right) \\
    D(t) &= \twopartdef{\frac{3}{4}(1-t^2)}{ \abs{t} \leq 1}{0}{\text{otherwise}}
\end{align*}
Now note that differentiating piecewise gives:
$$ \frac{\partial D(t)}{\partial t} = \twopartdef{-\frac{3}{2}t}{ \abs{t} \leq 1}{0}{\text{otherwise}} $$

In particular, taking limits as $t \rightarrow 1$ from above and below gives that $D(t) = -\frac{3}{2}$ and $D(t) = 0$ respectively, and so $D$ is not differentiable. 
For $t = \frac{\abs{x - x_0}}{\lambda(x_0)}$,we have:

$$\frac{\partial K_\lambda(x_0, x)}{\partial x_0} = \frac{\partial D}{\partial t} \frac{\partial t}{\partial x_0} $$
This will always have issues where $t = 1$, even if $\lambda$ is differentiable in $x_0$.

\subsection*{6.2 Show that $\sum_{i=1}^N (x_i - x_0) l_i(x_0) = 0$ for local linear regression.
Define $b_j(x_0) =  \sum_{i=1}^N (x_i - x_0)^j l_i(x_0)$. Show that $b_0(x_0) = 1$ for local polynomial regression of any degree. Show that $b_j(x_0) = 0$ for $0 < j \leq k$ for local polynomial regression of degree $k$. What are the implications of this on the bias?}

Let $l(x_0) = [l_1(x_0), \dots , l_N(x_0)]^T$. Then using the notation of subsection 6.1.1, in particular (6.8), we have:
$$ l(x_0)^T = b(x_0)^T \left(B^T W(x_0) B \right)^{-1} B^T W(x_0)$$
Where $W$ is the diagonal matrix of kernel weights (i.e. $K_\lambda(x_0, x_i)$ and $B$ being the $N\times (N+1)$ matrix with $i$th row $b(x_i)^T = [1,x,x^2,\dots,x^N]$

Thus 
\begin{align*}
    l(x_0)^T B &= [\sum_{i=1}^N l_i(x_0), \sum_{i=1}^N l_i(x_0) x_i , \dots, \sum_{i=1}^N l_i(x_0) x_i^N ] \\
    l(x_0)^T B &= b(x_0)^T \\
    &= [1, x_0, \dots, x_0^N]
\end{align*}

Comparing element-wise we see that $\sum_{i=1}^N l_i(x_0) = 1$ and $\sum_{i=1}^N l_i(x_0) x_i^j = x_0^j$.
Thus $\sum_{i=1}^N l_i(x_0) (x_i - x_0) = 0$ and $b_0(x_0) = 1$.
Lastly:
\begin{align*}
    b_j(x_0) &= \sum_{i=1}^N (x_i - x_0)^j l_i(x_0) \\
    &= \sum_{i=1}^N \sum_{k=0}^j {j \choose k} (-1) ^{j-k} x_0^{j-k} x_i^k l_i(x_0) \\
    &= \sum_{k=0}^j {j \choose k} (-1) ^{j-k} x_0^{j-k} \sum_{i=1}^N x_i^k l_i(x_0) \\ 
    &= \sum_{k=0}^j {j \choose k} (-1) ^{j-k} x_0^{j-k} x_0^k \\ 
    &= (x_0 - x_0)^j \\
    &= 0
\end{align*}

This suggests to bias is $0$ to order $k$ (i.e. if $y$ is really a polynomial in $x$ of degree at most $k$, the model is unbiased).

\subsection*{6.5 Show that fitting a locally constant multinomial logit model of the form (6.19) amounts to smoothing the binary response indicators for each class separately using a Nadaraya-Watson kernel smoother with kernel weights $K_\lambda(x_0, x_i)$}

(6.19) The local log-likelihood for class $j$ under a locally constant multinomial logit model is:
$$ \sum_{i=1}^N K_\lambda (x_0, x_i) \left( \beta_j(x_0) - \log\left( 1 + \sum_{j=1}^{J-1} \exp{\beta_j(x_0)}\right) \right)$$

In particular, writing $y_{i,j} = \delta_{y_i, g_j}$ for the function that is $1$ if the $i$th observation is in class $j$ and $0$ otherwise\footnote{This is an explicit form of the binary response indicator vector}, we can get:
$$l(\beta) = \sum_{i=1}^N K_\lambda (x_0, x_i) \left( \sum_{j=1}^{J-1} \beta_j(x_0) y_{i,j} - \log\left( 1 + \sum_{j=1}^{J-1} \exp{\beta_j(x_0)}\right) \right)$$

We also have:
$$p_j = \mathbb{P}(G = j \vert X = x) = \frac{\exp(\beta_j)}{1+\sum_{j=1}^{J-1} \exp(\beta_j}$$

Differentiating w.r.t $\beta_j$ and setting the result to $0$ we get:
\begin{align*}
   \frac{\partial l(\beta)}{\partial \beta_j} &= \sum_{i=1}^N K_\lambda(x_0, x_i) \left(y_{i,j} - \frac{\exp(\beta_j)}{1+\sum_{j=1}^{J-1} \exp(\beta_j}\right) \\
   &= \sum_{i=1}^N K_\lambda(x_0, x_i) \left(y_{i,j} - p_j\right) \\
   \Rightarrow p_j &= \frac{ \sum_{i=1}^N K_\lambda(x_0, x_i) y_{i,j} } {\sum_{i=1}^N K_\lambda(x_0, x_i) }
\end{align*} 
This condition for selecting the $\beta_j$ is exactly the result of smoothing each class separately under a Nadaraya-Watson kernel smoother.

\subsection*{6.7 Derive an expression for the leave-one-out cross-validated RSS for local polynomial regression}

$$\frac{1}{N} \sum_{i=1}^N \sum_{j \neq i} K_\lambda(x_j, x_i)^2 \norm{y-x_i^T\beta_j}^2$$

\subsection*{6.8 Suppose that for continuous response $Y$ and predictor $X$, we model the joint density of $X,Y$ using a multivariate Gaussian kernel estimator. Note that the kernel in this case would be the product kernel $\phi_\lambda(X)\phi_\lambda(Y)$. Show that the conditional mean $\mathbb{E}\left[Y\vert X\right]$ derived from this estimate is a Nadaraya-Watson estimator. Extend this result to classification by providing a suitable kernel for the estimation of the joint distribution of a continuous $X$ and discrete $Y$}

\begin{align*}
    \hat{f}_{X,Y}(x,y) &= \frac{1}{N} \sum_{i=1}^N \phi_\lambda(x-x_i)\phi_Y(y-y_i)\\
    \hat{f}_{X}(x) &= \frac{1}{N} \sum_{i=1}^N \phi_\lambda(x-x_i)
\end{align*}

Thus 

\begin{align*}
    \mathbb{E}\left[Y\vert X\right] 
    &= \int_{-\infty}^\infty \frac{\hat{f}_{X,Y}(x,y) y}{\hat{f}_{X}(x)} dy \\
    &= \int_{-\infty}^\infty \frac{\sum_{i=1}^N \phi_\lambda(x-x_i)\phi_\lambda(y-y_i) y }{\sum_{i=1}^N \phi_\lambda(x-x_i)} dy  \\
    &= \frac{\sum_{i=1}^N \phi_\lambda(x-x_i)\int_{-\infty}^\infty y \phi_\lambda(y-y_i)  dy} {\sum_{i=1}^N  \phi_\lambda(x-x_i)}\\
    &= \frac{\sum_{i=1}^N \phi_\lambda(x-x_i)\int_{-\infty}^\infty y \phi_\lambda(y-y_i)  dy} {\sum_{i=1}^N  \phi_\lambda(x-x_i)}\\
    \int_{-\infty}^\infty y \phi_\lambda(y-y_i)  dy
    &= \int_{-\infty}^\infty (y-y_i) \phi_\lambda(y-y_i)  dy + y_i \int_{-\infty}^\infty  \phi_\lambda(y-y_i)  dy \\
    &=  \mathbb{E}\left[Y\right]  + y_i \\
    &= y_i \\
\Rightarrow  \mathbb{E}\left[Y\vert X\right] &=\frac{\sum_{i=1}^N \phi_\lambda(x-x_i) y_i} {\sum_{i=1}^N  \phi_\lambda(x-x_i)}\\
\end{align*}
Thus the conditional expectation is a Nadaraya-Watson estimator.

For classification, let our kernel be $\phi_\lambda(X)\delta_g(Y)$ where:
$$\twopartdef{\delta_g(y) = 1}{y_i = g}{0}{\text{otherwise}}$$

That is $\hat{f}_{X, Y}(x,y) = \frac{1}{N} \sum_{i=1}^N \phi_\lambda(x-x_i) \delta_y(y_i)$. 
Then 
$$\mathbb{P}(Y=y|X=x) = \frac{\sum_{y_i=y} \phi_\lambda(x-x_i)} {\sum_{i=1}^N \phi_\lambda(x-x_i)}$$

This definition makes intuitive sense - in the case where we instead had $X$ as a constant function say, it would be the sample estimate. Once again the expected value becomes a Nadaraya-Watson estimator\footnote{Where we have $j$ classes for the response, $y_1, \dots, y_j$}:
\begin{align*}
    \mathbb{E}\left[Y\vert X\right] 
    &= \frac{\sum_{j = 1}^M \sum_{y_i=y_j} \phi_\lambda(x-x_i) y_i} {\sum_{i=1}^N \phi_\lambda(x-x_i)} \\
    &= \frac{\sum_{i = 1}^N \phi_\lambda(x-x_i) y_i} {\sum_{i=1}^N \phi_\lambda(x-x_i)} 
\end{align*}
