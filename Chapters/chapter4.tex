
\chapter{Chapter 4 - Linear methods for Classification}

\subsection*{4.1 Show how to solve the generalised eigenvalue problem
\begin{align*}
    max_a & a^T Ba\\
    \text{subject to }& a^T W a =1
\end{align*}
by transforming to a standard eigenvalue problem.}

Via Lagrange multipliers we have $D (a^T B a) = \lambda D (a^T W a -1)$ for some $\lambda$.
This gives:
\begin{align*}
& D (a^T B a) = \lambda D (a^T W a -1)\\
    &\Rightarrow 2 B a = 2\lambda W a \\ 
   &\Rightarrow B a = \lambda W a \\ 
   &\Rightarrow W^{-1}B a = \lambda a 
\end{align*}
So $a$ is an eigenvector of $W^{-1}B$ with eigenvalue $\lambda$. We would select the $a$ corresponding to the largest eigenvalue\footnote{We can always scale $a$ such that our constraint is satisfied.}, as we have
$a^T B a = \lambda a^T W a = \lambda$.

\subsection*{4.2 Suppose we have features $x \in \mathbb{R}^p$, a two class response, with class sizes $N_1, N_2$, and the target coded as $-N/N_1, N/N_2$.}
\subsubsection*{4.2a) Show that the LDA rule classifies to class 2 if 
$$ x^T \Sigma^{-1} (\hat{\mu}_2 - \hat{\mu}_1) > \frac{1}{2} (\hat{\mu}_2 + \hat{\mu}_1) \Sigma^{-1} (\hat{\mu}_2 - \hat{\mu}_1) - log(N_2 / N_1)$$
and class 1 otherwise
}
We have
$$log \frac{\mathbb{P}(G = 2 \vert X = x)}{\mathbb{P}(G = 1 \vert X = x)} = \log\left(\frac{\pi_2}{\pi_1}\right) - \frac{1}{2}(\hat{\mu}_2 + \hat{\mu}_1) \Sigma^{-1} (\hat{\mu}_2 - \hat{\mu}_1)  + x^T \Sigma^{-1} (\hat{\mu}_2 - \hat{\mu}_1)$$
and we classify to class 2 if this value is at at least $0$.
Expanding gives:
\begin{align*}
    &\frac{\mathbb{P}(G = 2 \vert X = x)}{\mathbb{P}(G = 1 \vert X = x)} > 0 \\
    &\iff \log\left(\frac{\pi_2}{\pi_1}\right) - \frac{1}{2}(\hat{\mu}_2 + \hat{\mu}_1) \Sigma^{-1} (\hat{\mu}_2 - \hat{\mu}_1)  + x^T \Sigma^{-1} (\hat{\mu}_2 - \hat{\mu}_1) > 0 \\
    &\iff \log\left(\frac{N_2}{N_1}\right) - \frac{1}{2}(\hat{\mu}_2 + \hat{\mu}_1) \Sigma^{-1} (\hat{\mu}_2 - \hat{\mu}_1)  + x^T \Sigma^{-1} (\hat{\mu}_2 - \hat{\mu}_1) > 0 \\
    &\iff x^T \Sigma^{-1} (\hat{\mu}_2 - \hat{\mu}_1) > \frac{1}{2}(\hat{\mu}_2 + \hat{\mu}_1) \Sigma^{-1} (\hat{\mu}_2 - \hat{\mu}_1) -\log\left(\frac{N_2}{N_1}\right)
\end{align*}

\subsubsection{b) Consider minimisation of the least squares criterion $\norm{y - \beta_0\cdot \mathbb{1} - X\beta}_2^2$. Show that the solution $\hat{\beta}$ satisfies
$$\left[ (N-2) \hat{\Sigma} + N \hat{\Sigma}_B \right] \beta = N (\hat{\mu}_2 - \hat{\mu}_1)$$
where $\hat{\Sigma}_B = \frac{N_1 N_2}{N^2}  (\hat{\mu}_2 - \hat{\mu}_1) (\hat{\mu}_2 - \hat{\mu}_1)^T$
}
Write $\mathbb{1}_i$ for the indicator vector corresponding to class $i$, so that $\mathbb{1} = \mathbb{1}_1 + \mathbb{1}_2$.
We will differentiate $\left(y - \beta_0\cdot \mathbb{1} - X\beta\right)^T\left(y - \beta_0\cdot \mathbb{1} - X\beta\right)$ w.r.t $\beta$ and $\beta_0$ and solve.
Setting the derivatives to zero gives:
\begin{align*}
    X^T (y - \hat{\beta}_0 \cdot \mathbb{1} - X \hat{\beta}) &= 0\\
    \mathbb{1}^T(y - \hat{\beta}_0 \cdot \mathbb{1} - X \hat{\beta}) &= 0
\end{align*}
Let $\Tilde{\mu} = (N_1 \hat{\mu}_1 + N_2 \hat{\mu}_2) = X^T \mathbb{1}$. Simplifying the above, we get.
\begin{align}
& X^T y - \frac{1}{N} X^T\mathbb{1} \mathbb{1}^T (y - X \hat{\beta}) = X^T X \hat{\beta}\\
    &\Rightarrow X^T y- \frac{1}{N} X^T\mathbb{1} \mathbb{1}^T y = X^T X \hat{\beta} - \frac{1}{N} X^T\mathbb{1} \mathbb{1}^T X  \hat{\beta} \\
    &\Rightarrow X^T y - \frac{1}{N} \Tilde{\mu} \mathbb{1}^T y = X^T X \hat{\beta} - \frac{1}{N} \Tilde{\mu} \Tilde{\mu}^T  \hat{\beta} \\
    &\Rightarrow X^T y - \frac{1}{N} \Tilde{\mu} (N_1 c_1 + N_2 c_2) = \left( X^T X - \frac{1}{N} \Tilde{\mu} \Tilde{\mu}^T \right) \hat{\beta} 
\end{align}

Expanding $\Tilde{\mu} \Tilde{\mu}^T$ gives:
\begin{align*}
\Tilde{\mu} \Tilde{\mu}^T &= 
(N_1 \hat{\mu}_1 + N_2 \hat{\mu}_2) (N_1 \hat{\mu}_1 + N_2 \hat{\mu}_2)^T \\
&= N_1^2 \hat{\mu}_1 \hat{\mu}_1^T + N_2^2 \hat{\mu}_2 \hat{\mu}_2^T + N_1 N_2(\hat{\mu}_1 \hat{\mu}_2^T + \hat{\mu}_2 \hat{\mu}_1^T)
\end{align*}

Now we want to establish the $\hat{\Sigma}$ terms
\begin{align*}
    \hat{\Sigma} &= \frac{1}{N-2} \left( \sum_{i = 1}^{N_1} (x_i - \hat{\mu}_1)(x_i - \hat{\mu}_1)^T + \sum_{i = N_11}^{N} (x_i - \hat{\mu}_2)(x_i - \hat{\mu}_2)^T + \right) \\ 
    &= \frac{1}{N-2} \left( \sum_{i = 1}^{N_1} (x_i - \hat{\mu}_1 x_i^T - x_i \hat{\mu}_1^T  + \hat{\mu}_1 \hat{\mu}_1^T )\right) \\ & + \frac{1}{N-2} \left(\sum_{i = N_1+1}^{N} (x_i - \hat{\mu}_2 x_i^T - x_i \hat{\mu}_2^T  + \hat{\mu}_2 \hat{\mu}_2^T ) \right) \\
    &= \frac{1}{N-2} \left( \sum_{i = 1}^{N} (x_i x_i^T)- 2  N_1 \hat{\mu}_1 \hat{\mu}_1^T + N_1\hat{\mu}_1 \hat{\mu}_1^T - 2 N_2 \hat{\mu}_2 \hat{\mu}_2^T  + N_2\hat{\mu}_2 \hat{\mu}_2^T \right) \\
    &= \frac{1}{N-2} \left( X^T X - N_1\hat{\mu}_1 \hat{\mu}_1^T - N_2\hat{\mu}_2 \hat{\mu}_2^T \right)
\end{align*}
This yields:
$$ X^T X = (N-2) \hat{\Sigma} + N_1\hat{\mu}_1 \hat{\mu}_1^T + N_2\hat{\mu}_2 \hat{\mu}_2^T $$
Thus:
\begin{align*}
 (N-2) \hat{\Sigma} + N \hat{\Sigma}_B &= X^T X - N_1\hat{\mu}_1 \hat{\mu}_1^T - N_2\hat{\mu}_2  \hat{\mu}_2^T + \frac{N_1 N_2}{N}  (\hat{\mu}_2 - \hat{\mu}_1) (\hat{\mu}_2 - \hat{\mu}_1)^T\\
 &= X^T X + \left(\frac{N_1 N_2}{N} -  N_1\right) \hat{\mu}_1 \hat{\mu}_1^T + \left(\frac{N_1 N_2}{N} -  N_2\right)\hat{\mu}_2 \hat{\mu}_2^T \\ &- \frac{N_1 N_2}{N}  (\hat{\mu}_1 \hat{\mu}_2^T - \hat{\mu}_2 \hat{\mu}_1^T)\\
  &= X^T X - \frac{N_1^2}{N} \hat{\mu}_1 \hat{\mu}_1^T - \frac{N_2^2}{N}\hat{\mu}_2 \hat{\mu}_2^T - \frac{N_1 N_2}{N}  (\hat{\mu}_1 \hat{\mu}_2^T - \hat{\mu}_2 \hat{\mu}_1^T)\\ 
  &= X^T X - \frac{1}{N}\Tilde{\mu} \Tilde{\mu}^T
\end{align*}
This is one side of equation $(15)$

We also know that:
$$X^T y = X^T(c_1 \mathbb{1}_1 + c_2 \mathbb{1}_2) = c_1 N_1 \mu_1 + c_2  N_2 \mu_2$$
And
\begin{align*}
     \frac{1}{N} \Tilde{\mu} (N_1 c_1 + N_2 c_2) &= 
\frac{\left(c_1 N_1^2 + c_2 N_1 N_2\right)}{N}\hat{\mu}_1 + \frac{\left(c_1 N_1 N_2 + c_2 N_2^2\right)}{N}\hat{\mu}_2 \\
&= \frac{\left(c_1 N_1 (N - N_2) +  c_2 N_1 N_2\right)}{N}\hat{\mu}_1 + \frac{\left(c_1 N_1 N_2 + c_2 N_2(N - N_1)\right)}{N}\hat{\mu}_2\\
&= \frac{\left(c_1 N_1 (N - N_2) +  c_2 N_1 N_2\right)}{N}\hat{\mu}_1 + \frac{\left(c_1 N_1 N_2 + c_2 N_2(N - N_1)\right)}{N}\hat{\mu}_2\\
&= c_1 N_1 \hat{\mu}_1 + c_2 N_2 \hat{\mu}_2 + \frac{N_1 N_2}{N} \left(\left(-c_1 +  c_2 \right)\hat{\mu}_1 + \left(c_1 - c_2 )\right)\hat{\mu}_2\right)\\
&= c_1 N_1 \hat{\mu}_1 + c_2 N_2 \hat{\mu}_2 + \frac{N_1 N_2}{N}(c_2 - c_1)(\hat{\mu}_1 - \hat{\mu}_2)
\end{align*}
Then 
$$X^T y - \frac{1}{N} \Tilde{\mu} (N_1 c_1 + N_2 c_2) = \frac{N_1 N_2}{N}(c_2 - c_1)(\hat{\mu}_1 - \hat{\mu}_2)$$

Thus we have all components of our above equation $(15)$

Combining and simplifying gives
$$(N-2) \hat{\Sigma} + N \hat{\Sigma}_B= \frac{N_1 N_2}{N}(c_2 - c_1)(\hat{\mu}_1 - \hat{\mu}_2) $$

Substituting in our values for $c_1$ and $c_2$ gives
$(N-2) \hat{\Sigma} + N \hat{\Sigma}_B= N(\hat{\mu}_1 - \hat{\mu}_2)  $ as required.



\subsubsection*{c) Hence show that $\hat{\Sigma}_B \beta$ is in the direction  $\hat{\mu}_2 - \hat{\mu}_1$ and thus
$$ \hat{\beta} \propto \hat{\Sigma}^{-1} \hat{\mu}_2 - \hat{\mu}_1 $$
}
For the direction:
\begin{align*}
    \hat{\Sigma}_B \beta &=  \frac{N_1 N_2}{N^2}  (\hat{\mu}_2 - \hat{\mu}_1) (\hat{\mu}_2 - \hat{\mu}_1)^T \beta \\
    &=  \frac{N_1 N_2}{N^2}  (\hat{\mu}_2 - \hat{\mu}_1) \cdot \lambda \text{ where } \lambda \in \mathbb{R} \\
    &=  \lambda' (\hat{\mu}_2 - \hat{\mu}_1) 
\end{align*}
For proportionality:
\begin{align*}
    &\left[ (N-2) \hat{\Sigma} + N \hat{\Sigma}_B \right] \hat{\beta} = N (\hat{\mu}_2 - \hat{\mu}_1) \\ 
    &\Rightarrow (N-2) \hat{\Sigma} \hat{\beta} = N (\hat{\mu}_2 - \hat{\mu}_1 - \hat{\Sigma}_B \hat{\beta} ) \\ 
    &\Rightarrow (N-2) \hat{\Sigma} \hat{\beta} = N (\hat{\mu}_2 - \hat{\mu}_1 - \hat{\Sigma}_B \hat{\beta} ) \\ 
    &\Rightarrow (N-2) \hat{\Sigma} \hat{\beta} = N (\hat{\mu}_2 - \hat{\mu}_1 -  \lambda' (\hat{\mu}_2 - \hat{\mu}_1) ) \\ 
    &\Rightarrow \hat{\Sigma} \hat{\beta} \propto \hat{\mu}_2 - \hat{\mu}_1 \\ 
    &\Rightarrow  \hat{\beta} \propto \hat{\Sigma}^{-1} \left(\hat{\mu}_2 - \hat{\mu}_1\right) \\ 
\end{align*}

\subsubsection*{d) Show that c) holds for any (distinct) coding of the two classes.}

This is simply the observation that $\frac{N_1 N_2}{N}(c_2 - c_1)$ is a scalar for any distinct coding.

\subsubsection*{ e) Find the solution $\hat{\beta}_0$ up to the same scalar multiple as in c). Hence find the solution for the predicted value $\hat{f}(x) = \hat{\beta}_0 + x^T \hat{\beta}$.
Consider the rule where we classify to class $2$ if $\hat{f}(x) > 0$ and class $1$ otherwise. Is this the same as the LDA rule? When?
}

\begin{align*}
    \hat{\beta}_0 &= \frac{1}{N}\mathbb{1}^T\left(y - X\hat{\beta}\right)\\
    &= \frac{N_1 c_1 + N_2 c_2}{N} - \frac{\lambda}{N} \left(N_1 \hat{\mu}_1 + N_2 \hat{\mu}_2\right)^T \Sigma^{-1} \left(\hat{\mu}_2 - \hat{\mu}_1 \right)\\
\end{align*}
Where $\lambda$ is the constant of proportionality from c).
Using our encoding, we get 
$$\hat{\beta}_0 =  - \frac{\lambda}{N} \left(N_1 \hat{\mu}_1 + N_2 \hat{\mu}_2\right)^T \Sigma^{-1} \left(\hat{\mu}_2 - \hat{\mu}_1 \right)$$

Then $$\hat{f}(x) = \lambda \left(x -\frac{N_1}{N} \hat{\mu}_1 - \frac{N_2}{N} \hat{\mu}_2\right)^T \Sigma^{-1} \left(\hat{\mu}_2 - \hat{\mu}_1 \right)$$

The classification is class $2$ when:
$$x^T \Sigma^{-1} \left(\hat{\mu}_2 - \hat{\mu}_1 \right) > \left(\frac{N_1}{N} \hat{\mu}_1 + \frac{N_2}{N} \hat{\mu}_2\right)^T \Sigma^{-1} \left(\hat{\mu}_2 - \hat{\mu}_1 \right)$$

This is the same as the LDA classification when $N_1 = N_2$, but in general is different.\footnote{
$N_1 = N_2$ gives $log (N_2 / N_1) = 0$ and $N_1 / N = N_2 / N = 1/2$ so the classification rules become identical. To see the difference set $\mu_1 = 0$ say, and the result should be clear. The log term depends only on the relative number of samples in each class, whereas the expression above depends on the average of those samples.}

\subsection*{4.3 Suppose we transform the original predictors $X$ to $\hat{Y}$ via linear regression.
$$\hat{Y} = X\hat{B} = X(X^T X)^{-1} X^T Y$$
where $Y$ is the indicator response matrix. Similarly for any input $x \in \mathbb{R}^p$ we get
$\hat{y} = = \hat{B}^T x \in \mathbb{R}^K$. Show that the LDA using $\hat{Y}$ is identical to the LDA in the original space.}

The discriminant functions for LDA are:
$$\delta_k(x) = x^T \hat{\Sigma}_X^{-1} \mu_k - \frac{1}{2} \mu_k^T \hat{\Sigma}_X^{-1}\mu_k + log \pi_k$$

The discriminant rule using our transformed predictors is (denoting the mean by $\eta$):
$$\delta_k(\hat{y}) = \hat{y}^T \hat{\Sigma}_{\hat{Y}}^{-1} \eta_k - \frac{1}{2} \eta_k^T \hat{\Sigma}_{\hat{Y}}^{-1}\eta_k + log \pi_k$$

Where $\eta_k = \hat{B}^T \mu_k$.


A few things to note:
\begin{align*}
    \mu &= X^T YD^{-1}\\
    \Sigma_X &= \frac{1}{N-K}\left(X^T - X^T Y D^{-1} Y^T\right) \left(X^T - X^T Y D^{-1} Y^T\right)^T\\
    &= \frac{1}{N-K} X^T \left(I - Y D^{-1} Y^T\right)^2 X\\
    &= \frac{1}{N-K} X^T \left(I - Y D^{-1} Y^T\right) X 
\end{align*}
Where $D$ is the $K \times K$ diagonal matrix with $D_{i,i} = N_i$, the number of observations in class $I$, and the last line follows as $(I - Y D^{-1} Y^T)^2 = (I - 2 Y D^{-1} Y^T + Y D^{-1} Y^T Y D^{-1} Y^T)$ and $Y^T Y = D$

\begin{align*}
    \hat{\Sigma}_{\hat{Y}} &= \frac{1}{N-K}\sum_{k = 1}^K \sum_{g_i = k} (\hat{y}_i - \eta_i) (\hat{y}_i - \eta_i)^T\\
    &= \frac{1}{N-K}\sum_{k = 1}^K \sum_{g_i = k} \hat{B}^T(x_i - \mu_i) (x_i - \mu_i)^T \hat{B}\\
    &=\hat{B}^T\left( \frac{1}{N-K}\sum_{k = 1}^K \sum_{g_i = k} (x_i - \mu_i) (x_i - \mu_i)^T \right)\hat{B}\\ 
    &=\hat{B}^T \hat{\Sigma}_X\hat{B}\\
    &=\frac{1}{N-K} \hat{B}^T X^T \left(I - Y D^{-1} Y^T\right) X\hat{B}\\
    &=\frac{1}{N-K} \left(\hat{B}^T X^T X\hat{B} - \hat{B}^T X^T Y D^{-1} Y^T X\hat{B}\right) \\
    &=\frac{1}{N-K} \left(\hat{B}^T X^T Y - \hat{B}^T X^T Y D^{-1} Y^T X\hat{B}\right) \\
    &=\frac{1}{N-K} \left(\Lambda - \Lambda D^{-1}\Lambda \right) \\
    &=\frac{1}{N-K} \Lambda\left(I - D^{-1}\Lambda \right) \\
\end{align*}
Where $\Lambda = \hat{B}^T X^T Y = Y^T X (X^T X)^{-1} X^T Y$ is symmetric.

We will now consider each term of our discriminant function, vectorised to handle all cases at once.
\begin{align*}
    \hat{Y}^T \hat{\Sigma}_{\hat{Y}}^{-1} \eta &=  X^T \hat{B}  \left(\hat{B}^T \hat{\Sigma}_X\hat{B}\right)^{-1} \hat{B} \mu\\ 
    &=  (N-K) 
    X^T \hat{B}  \left(\Lambda - \Lambda D^{-1}\Lambda \right)^{-1} \hat{B}^T X^T Y D^{-1}\\
    &=  (N-K) X^T \hat{B}  \left(\Lambda - \Lambda D^{-1}\Lambda \right)^{-1} \Lambda D^{-1}\\
    &=  (N-K) X^T \hat{B}  \left(I - D^{-1}\Lambda \right)^{-1}  D^{-1}\\
    &=  (N-K) X^T \Sigma_X^{-1} \Sigma_X \hat{B}  \left(I - D^{-1}\Lambda \right)^{-1}  D^{-1}\\
    &=  X^T \Sigma_X^{-1} \left( X^T X \hat{B} - X^T Y D^{-1} Y^T X \hat{B}\right)   \left(I - D^{-1}\Lambda \right)^{-1}  D^{-1}\\
    &=  X^T \Sigma_X^{-1} \left( X^T Y - X^T Y D^{-1} \Lambda \right)   \left(I - D^{-1}\Lambda \right)^{-1}  D^{-1}\\
    &=  X^T \Sigma_X^{-1} X^T Y \left( I - D^{-1} \Lambda \right)   \left(I - D^{-1}\Lambda \right)^{-1}  D^{-1}\\
    &=  X^T \Sigma_X^{-1} X^T Y   D^{-1}\\
    &=  X^T \Sigma_X^{-1} \mu\\
\end{align*}
So the first terms of our discriminant functions are equal. For the second term, again dealing with all classes at once:

\begin{align*}
    \eta_k^T \hat{\Sigma}_{\hat{Y}}^{-1}\eta &= \left(N-K\right) \mu_k^T \hat{B} \hat{\Sigma}_{\hat{Y}}^{-1} \hat{B}^T X^T Y D^{-1}\\
    &=  \mu_k^T \hat{B} \left(\hat{B}^T\hat{\Sigma}_{X} \hat{B} \right)^{-1} \hat{B}^T X^T Y D^{-1}\\
    &= \mu_k^T \hat{\Sigma}_{X}^{-1} X^T Y D^{-1}\\
     &= \mu_k^T \hat{\Sigma}_{X}^{-1} \mu\\
\end{align*}
Where the penultimate line follows by the same algebra as used for the first term of our discriminant functions\footnote{Noting that when handling $\hat{Y}^T \hat{\Sigma}_{\hat{Y}}^{-1} \eta$ we could have dropped the leading $X^T$ term throughout and dealt with $\hat{B} \hat{\Sigma}_{\hat{Y}}^{-1} \eta$ instead.}
The third term in each discriminant function is unchanged by our transformation as it depends only on the response. Thus all terms are equal and so the LDA on $\hat{Y}$ is equivalent to the LDA on $X$.


\subsection*{ 4.6 Convergence of the perceptron algorithm.}
\subsection{a) Please see textbook - existence of a $\beta_sep$ such that $y_i \beta_{sep}^T z_i >= 1$ for all $i$ given separable data.}

The data are separable so $\exists \beta$ s.t. $ sign(\beta^T x_i^*) = y_i$.
Thus we have $v_i := y_i \beta^T z_i > 0$ for every $i$.
There are finitely many $i$, so we can set $\beta_{sep} = \beta / min_i v_i$ to get $y_i \beta_{sep}^T z_i >= 1$ for all $i$.

\subsection{b) Please see the textbook - convergence of the perceptron.}

We update $\beta$ via $\beta_{new} \leftarrow \beta_{old} + y_i z_i$

\begin{align*}
    \norm{\beta_{new} - \beta_{sep}}^2 &= \norm{\beta_{old} + y_i z_i - \beta_{sep}}^2\\
    &= \left(\beta_{old} + y_i z_i - \beta_{sep}\right)^T\left(\beta_{old} + y_i z_i - \beta_{sep}\right)\\
    &= \norm{\beta_{old}  - \beta_{sep}}^2 - 2 y_i \beta_{sep}^T z_i + 2 y_i \beta_{old}^T z_i + y_i^2 z_i^T z_i \\
    &= \norm{\beta_{old}  - \beta_{sep}}^2 + 2 y_i \left(\beta_{old} -\beta_{sep}\right)^T z_i + \norm{z}_i^2\\
    &= \norm{\beta_{old}  - \beta_{sep}}^2 + 2 y_i \left(\beta_{old} -\beta_{sep}\right)^T z_i + 1\\
    &<= \norm{\beta_{old}  - \beta_{sep}}^2 - 2 + 1\\
    &= \left(\beta_{old}  - \beta_{sep}\right)^T\left(\beta_{old}  - \beta_{sep}\right) - 1
\end{align*}
Where the penultimate line follows as $\beta_{old}$ misclassifies $z_i$ - that is to say
$y_i \beta_{old} z_i < 0$. Hence the term $y_i \left(\beta_{old} -\beta_{sep}\right)^Tz_i$ is at most $- y_i \beta_{sep}^T z_i$, and we use part a).

Hence the perceptron converges in at most $norm{\beta_{old}  - \beta_{sep}}^2 $ steps.
