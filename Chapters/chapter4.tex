
\chapter{Chapter 4 - Linear methods for Classification}

\subsection*{4.1 Show how to solve the generalised eigenvalue problem
\begin{align*}
    max_a & a^T Ba\\
    \text{subject to }& a^T W a =1
\end{align*}
by transforming to a standard eigenvalue problem.}

Via Lagrange multipliers we have $D (a^T B a) = \lambda D (a^T W a -1)$ for some $\lambda$.
This gives:
\begin{align*}
& D (a^T B a) = \lambda D (a^T W a -1)\\
    &\Rightarrow 2 B a = 2\lambda W a \\ 
   &\Rightarrow B a = \lambda W a \\ 
   &\Rightarrow W^{-1}B a = \lambda a 
\end{align*}
So $a$ is an eigenvector of $W^{-1}B$ with eigenvalue $\lambda$. We would select the $a$ corresponding to the largest eigenvalue\footnote{We can always scale $a$ such that our constraint is satisfied.}, as we have
$a^T B a = \lambda a^T W a = \lambda$.

\subsection*{4.2 Suppose we have features $x \in \mathbb{R}^p$, a two class response, with class sizes $N_1, N_2$, and the target coded as $-N/N_1, N/N_2$.}
\subsubsection*{4.2a) Show that the LDA rule classifies to class 2 if 
$$ x^T \Sigma^{-1} (\hat{\mu}_2 - \hat{\mu}_1) > \frac{1}{2} (\hat{\mu}_2 + \hat{\mu}_1) \Sigma^{-1} (\hat{\mu}_2 - \hat{\mu}_1) - log(N_2 / N_1)$$
and class 1 otherwise
}
We have
$$log \frac{\mathbb{P}(G = 2 \vert X = x)}{\mathbb{P}(G = 1 \vert X = x)} = \log\left(\frac{\pi_2}{\pi_1}\right) - \frac{1}{2}(\hat{\mu}_2 + \hat{\mu}_1) \Sigma^{-1} (\hat{\mu}_2 - \hat{\mu}_1)  + x^T \Sigma^{-1} (\hat{\mu}_2 - \hat{\mu}_1)$$
and we classify to class 2 if this value is at at least $0$.
Expanding gives:
\begin{align*}
    &\frac{\mathbb{P}(G = 2 \vert X = x)}{\mathbb{P}(G = 1 \vert X = x)} > 0 \\
    &\iff \log\left(\frac{\pi_2}{\pi_1}\right) - \frac{1}{2}(\hat{\mu}_2 + \hat{\mu}_1) \Sigma^{-1} (\hat{\mu}_2 - \hat{\mu}_1)  + x^T \Sigma^{-1} (\hat{\mu}_2 - \hat{\mu}_1) > 0 \\
    &\iff \log\left(\frac{N_2}{N_1}\right) - \frac{1}{2}(\hat{\mu}_2 + \hat{\mu}_1) \Sigma^{-1} (\hat{\mu}_2 - \hat{\mu}_1)  + x^T \Sigma^{-1} (\hat{\mu}_2 - \hat{\mu}_1) > 0 \\
    &\iff x^T \Sigma^{-1} (\hat{\mu}_2 - \hat{\mu}_1) > \frac{1}{2}(\hat{\mu}_2 + \hat{\mu}_1) \Sigma^{-1} (\hat{\mu}_2 - \hat{\mu}_1) -\log\left(\frac{N_2}{N_1}\right)
\end{align*}

\subsubsection{b) Consider minimisation of the least squares criterion $\norm{y - \beta_0\cdot \mathbb{1} - X\beta}_2^2$. Show that the solution $\hat{\beta}$ satisfies
$$\left[ (N-2) \hat{\Sigma} + N \hat{\Sigma}_B \right] \beta = N (\hat{\mu}_2 - \hat{\mu}_1)$$
where $\hat{\Sigma}_B = \frac{N_1 N_2}{N^2}  (\hat{\mu}_2 - \hat{\mu}_1) (\hat{\mu}_2 - \hat{\mu}_1)^T$
}
Write $\mathbb{1}_i$ for the indicator vector corresponding to class $i$, so that $\mathbb{1} = \mathbb{1}_1 + \mathbb{1}_2$.
We will differentiate $\left(y - \beta_0\cdot \mathbb{1} - X\beta\right)^T\left(y - \beta_0\cdot \mathbb{1} - X\beta\right)$ w.r.t $\beta$ and $\beta_0$ and solve.
Setting the derivatives to zero gives:
\begin{align*}
    X^T (y - \hat{\beta}_0 \cdot \mathbb{1} - X \hat{\beta}) &= 0\\
    \mathbb{1}^T(y - \hat{\beta}_0 \cdot \mathbb{1} - X \hat{\beta}) &= 0
\end{align*}
Let $\Tilde{\mu} = (N_1 \hat{\mu}_1 + N_2 \hat{\mu}_2) = X^T \mathbb{1}$. Simplifying the above, we get.
\begin{align}
& X^T y - \frac{1}{N} X^T\mathbb{1} \mathbb{1}^T (y - X \hat{\beta}) = X^T X \hat{\beta}\\
    &\Rightarrow X^T y- \frac{1}{N} X^T\mathbb{1} \mathbb{1}^T y = X^T X \hat{\beta} - \frac{1}{N} X^T\mathbb{1} \mathbb{1}^T X  \hat{\beta} \\
    &\Rightarrow X^T y - \frac{1}{N} \Tilde{\mu} \mathbb{1}^T y = X^T X \hat{\beta} - \frac{1}{N} \Tilde{\mu} \Tilde{\mu}^T  \hat{\beta} \\
    &\Rightarrow X^T y - \frac{1}{N} \Tilde{\mu} (N_1 c_1 + N_2 c_2) = \left( X^T X - \frac{1}{N} \Tilde{\mu} \Tilde{\mu}^T \right) \hat{\beta} 
\end{align}

Expanding $\Tilde{\mu} \Tilde{\mu}^T$ gives:
\begin{align*}
\Tilde{\mu} \Tilde{\mu}^T &= 
(N_1 \hat{\mu}_1 + N_2 \hat{\mu}_2) (N_1 \hat{\mu}_1 + N_2 \hat{\mu}_2)^T \\
&= N_1^2 \hat{\mu}_1 \hat{\mu}_1^T + N_2^2 \hat{\mu}_2 \hat{\mu}_2^T + N_1 N_2(\hat{\mu}_1 \hat{\mu}_2^T + \hat{\mu}_2 \hat{\mu}_1^T)
\end{align*}

Now we want to establish the $\hat{\Sigma}$ terms
\begin{align*}
    \hat{\Sigma} &= \frac{1}{N-2} \left( \sum_{i = 1}^{N_1} (x_i - \hat{\mu}_1)(x_i - \hat{\mu}_1)^T + \sum_{i = N_11}^{N} (x_i - \hat{\mu}_2)(x_i - \hat{\mu}_2)^T + \right) \\ 
    &= \frac{1}{N-2} \left( \sum_{i = 1}^{N_1} (x_i - \hat{\mu}_1 x_i^T - x_i \hat{\mu}_1^T  + \hat{\mu}_1 \hat{\mu}_1^T )\right) \\ & + \frac{1}{N-2} \left(\sum_{i = N_1+1}^{N} (x_i - \hat{\mu}_2 x_i^T - x_i \hat{\mu}_2^T  + \hat{\mu}_2 \hat{\mu}_2^T ) \right) \\
    &= \frac{1}{N-2} \left( \sum_{i = 1}^{N} (x_i x_i^T)- 2  N_1 \hat{\mu}_1 \hat{\mu}_1^T + N_1\hat{\mu}_1 \hat{\mu}_1^T - 2 N_2 \hat{\mu}_2 \hat{\mu}_2^T  + N_2\hat{\mu}_2 \hat{\mu}_2^T \right) \\
    &= \frac{1}{N-2} \left( X^T X - N_1\hat{\mu}_1 \hat{\mu}_1^T - N_2\hat{\mu}_2 \hat{\mu}_2^T \right)
\end{align*}
This yields:
$$ X^T X = (N-2) \hat{\Sigma} + N_1\hat{\mu}_1 \hat{\mu}_1^T + N_2\hat{\mu}_2 \hat{\mu}_2^T $$
Thus:
\begin{align*}
 (N-2) \hat{\Sigma} + N \hat{\Sigma}_B &= X^T X - N_1\hat{\mu}_1 \hat{\mu}_1^T - N_2\hat{\mu}_2  \hat{\mu}_2^T + \frac{N_1 N_2}{N}  (\hat{\mu}_2 - \hat{\mu}_1) (\hat{\mu}_2 - \hat{\mu}_1)^T\\
 &= X^T X + \left(\frac{N_1 N_2}{N} -  N_1\right) \hat{\mu}_1 \hat{\mu}_1^T + \left(\frac{N_1 N_2}{N} -  N_2\right)\hat{\mu}_2 \hat{\mu}_2^T \\ &- \frac{N_1 N_2}{N}  (\hat{\mu}_1 \hat{\mu}_2^T - \hat{\mu}_2 \hat{\mu}_1^T)\\
  &= X^T X - \frac{N_1^2}{N} \hat{\mu}_1 \hat{\mu}_1^T - \frac{N_2^2}{N}\hat{\mu}_2 \hat{\mu}_2^T - \frac{N_1 N_2}{N}  (\hat{\mu}_1 \hat{\mu}_2^T - \hat{\mu}_2 \hat{\mu}_1^T)\\ 
  &= X^T X - \frac{1}{N}\Tilde{\mu} \Tilde{\mu}^T
\end{align*}
This is one side of equation $(15)$

We also know that:
$$X^T y = X^T(c_1 \mathbb{1}_1 + c_2 \mathbb{1}_2) = c_1 N_1 \mu_1 + c_2  N_2 \mu_2$$
And
\begin{align*}
     \frac{1}{N} \Tilde{\mu} (N_1 c_1 + N_2 c_2) &= 
\frac{\left(c_1 N_1^2 + c_2 N_1 N_2\right)}{N}\hat{\mu}_1 + \frac{\left(c_1 N_1 N_2 + c_2 N_2^2\right)}{N}\hat{\mu}_2 \\
&= \frac{\left(c_1 N_1 (N - N_2) +  c_2 N_1 N_2\right)}{N}\hat{\mu}_1 + \frac{\left(c_1 N_1 N_2 + c_2 N_2(N - N_1)\right)}{N}\hat{\mu}_2\\
&= \frac{\left(c_1 N_1 (N - N_2) +  c_2 N_1 N_2\right)}{N}\hat{\mu}_1 + \frac{\left(c_1 N_1 N_2 + c_2 N_2(N - N_1)\right)}{N}\hat{\mu}_2\\
&= c_1 N_1 \hat{\mu}_1 + c_2 N_2 \hat{\mu}_2 + \frac{N_1 N_2}{N} \left(\left(-c_1 +  c_2 \right)\hat{\mu}_1 + \left(c_1 - c_2 )\right)\hat{\mu}_2\right)\\
&= c_1 N_1 \hat{\mu}_1 + c_2 N_2 \hat{\mu}_2 + \frac{N_1 N_2}{N}(c_2 - c_1)(\hat{\mu}_1 - \hat{\mu}_2)
\end{align*}
Then 
$$X^T y - \frac{1}{N} \Tilde{\mu} (N_1 c_1 + N_2 c_2) = \frac{N_1 N_2}{N}(c_2 - c_1)(\hat{\mu}_1 - \hat{\mu}_2)$$

Thus we have all components of our above equation $(15)$

Combining and simplifying gives
$$(N-2) \hat{\Sigma} + N \hat{\Sigma}_B= \frac{N_1 N_2}{N}(c_2 - c_1)(\hat{\mu}_1 - \hat{\mu}_2) $$

Substituting in our values for $c_1$ and $c_2$ gives
$(N-2) \hat{\Sigma} + N \hat{\Sigma}_B= N(\hat{\mu}_1 - \hat{\mu}_2)  $ as required.



\subsubsection*{c) Hence show that $\hat{\Sigma}_B \beta$ is in the direction  $\hat{\mu}_2 - \hat{\mu}_1$ and thus
$$ \hat{\beta} \propto \hat{\Sigma}^{-1} \hat{\mu}_2 - \hat{\mu}_1 $$
}
For the direction:
\begin{align*}
    \hat{\Sigma}_B \beta &=  \frac{N_1 N_2}{N^2}  (\hat{\mu}_2 - \hat{\mu}_1) (\hat{\mu}_2 - \hat{\mu}_1)^T \beta \\
    &=  \frac{N_1 N_2}{N^2}  (\hat{\mu}_2 - \hat{\mu}_1) \cdot \lambda \text{ where } \lambda \in \mathbb{R} \\
    &=  \lambda' (\hat{\mu}_2 - \hat{\mu}_1) 
\end{align*}
For proportionality:
\begin{align*}
    &\left[ (N-2) \hat{\Sigma} + N \hat{\Sigma}_B \right] \hat{\beta} = N (\hat{\mu}_2 - \hat{\mu}_1) \\ 
    &\Rightarrow (N-2) \hat{\Sigma} \hat{\beta} = N (\hat{\mu}_2 - \hat{\mu}_1 - \hat{\Sigma}_B \hat{\beta} ) \\ 
    &\Rightarrow (N-2) \hat{\Sigma} \hat{\beta} = N (\hat{\mu}_2 - \hat{\mu}_1 - \hat{\Sigma}_B \hat{\beta} ) \\ 
    &\Rightarrow (N-2) \hat{\Sigma} \hat{\beta} = N (\hat{\mu}_2 - \hat{\mu}_1 -  \lambda' (\hat{\mu}_2 - \hat{\mu}_1) ) \\ 
    &\Rightarrow \hat{\Sigma} \hat{\beta} \propto \hat{\mu}_2 - \hat{\mu}_1 \\ 
    &\Rightarrow  \hat{\beta} \propto \hat{\Sigma}^{-1} \left(\hat{\mu}_2 - \hat{\mu}_1\right) \\ 
\end{align*}

\subsubsection*{d) Show that c) holds for any (distinct) coding of the two classes.}

This is simply the observation that $\frac{N_1 N_2}{N}(c_2 - c_1)$ is a scalar for any distinct coding.

\subsubsection*{ e) Find the solution $\hat{\beta}_0$ up to the same scalar multiple as in c). Hence find the solution for the predicted value $\hat{f}(x) = \hat{\beta}_0 + x^T \hat{\beta}$.
Consider the rule where we classify to class $2$ if $\hat{f}(x) > 0$ and class $1$ otherwise. Is this the same as the LDA rule? When?
}

\begin{align*}
    \hat{\beta}_0 &= \frac{1}{N}\mathbb{1}^T\left(y - X\hat{\beta}\right)\\
    &= \frac{N_1 c_1 + N_2 c_2}{N} - \frac{\lambda}{N} \left(N_1 \hat{\mu}_1 + N_2 \hat{\mu}_2\right)^T \Sigma^{-1} \left(\hat{\mu}_2 - \hat{\mu}_1 \right)\\
\end{align*}
Where $\lambda$ is the constant of proportionality from c).
Using our encoding, we get 
$$\hat{\beta}_0 =  - \frac{\lambda}{N} \left(N_1 \hat{\mu}_1 + N_2 \hat{\mu}_2\right)^T \Sigma^{-1} \left(\hat{\mu}_2 - \hat{\mu}_1 \right)$$

Then $$\hat{f}(x) = \lambda \left(x -\frac{N_1}{N} \hat{\mu}_1 - \frac{N_2}{N} \hat{\mu}_2\right)^T \Sigma^{-1} \left(\hat{\mu}_2 - \hat{\mu}_1 \right)$$

The classification is class $2$ when:
$$x^T \Sigma^{-1} \left(\hat{\mu}_2 - \hat{\mu}_1 \right) > \left(\frac{N_1}{N} \hat{\mu}_1 + \frac{N_2}{N} \hat{\mu}_2\right)^T \Sigma^{-1} \left(\hat{\mu}_2 - \hat{\mu}_1 \right)$$

This is the same as the LDA classification when $N_1 = N_2$, but in general is different.\footnote{
$N_1 = N_2$ gives $log (N_2 / N_1) = 0$ and $N_1 / N = N_2 / N = 1/2$ so the classification rules become identical. To see the difference set $\mu_1 = 0$ say, and the result should be clear. The log term depends only on the relative number of samples in each class, whereas the expression above depends on the average of those samples.}